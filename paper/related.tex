\section{Related Work}
\label{sec:related}

In 1995 Green first described the addition of Reversible jumps to the Markov Chain Monte Carlo method for Bayesian
computation, allowing the use of varying dimensionality parameter vectors \cite{green1995reversible}. The algorithm has
applicability to a range of fields, however, we are interested in the application of RJ-MCMC to multiple agent tracking in computer
vision.

In 2004 T. Zhao et al. \cite{zhao2004tracking} released a general framework based on the RJ-MCMC algorithm for tracking
multiple humans in crowded environments. Instead of using traditional image segmentation, which results in a large number of
parameters, the framework relied on minimally characterized shape models. Yet, the resulting framework could only achieve a
frame rate of about 3fps.

In 2010 Choi et al. \cite{choi2010multiple} developed a multiple agent tracking implementation framework also based on the RJ
MCMC algorithm. The aim of the framework was to create an interchangeable foundation, where elements such as the detection
and the tracking scheme could easily be swapped out in favor of improved versions as they are developed. Yet, the resulting
framework required extensive pre-processing and was computationally expensive.

Both \cite{zhao2004tracking}'s and \cite{choi2010multiple}'s frameworks suffer from an inability to function in real-time. However,
within this paper we are interested in determining the applicability of \cite{choi2010multiple}'s framework to a real-time tracking
system as we have previously dealt with this framework. Our work was intended to contribute to \cite{choi2010multiple}'s
framework by determining the extensibility of the foundation they laid. We analyzed and profiled the original framework, and
determined the slowest step was the detector being used. To bypass this issue we replaced their detector with our own, real time
detector, designed to detect simpler objects. This allowed us to profile the tracking framework in a realistic situation and gain a
better understanding of the practicality of the framework.