\section{Implementation}
\label{sec:impl}

This section gives an overview of how our changes to the original system were implemented. The main change was replacing the
original observation node, which read the results from the MATLAB preprocessing as described above, with our blob detector
node. While some minor changes were made to the parameters set and initializing the node, as little code was modified as
possible to give a consistent comparison to the original system. As such no other optimizations were made.

\subsection{MSER Blob Detection}
In order to quickly reduce the search space and eliminate a large amount of needless noise from the raw image data a few
pre-processing steps are applied before the detector. The sequence of applied steps is illustrated in Figure TODO.

The first step we took was to make an estimate of the horizon line within the image. We then discarded all image data above the
horizon estimate under the assumption that the Spheros would lie on the ground plane, below the horizon line. Next we
converted the image data from the RGB color-space to HSV, which proved to be a more robust color-space when dealing with
the self illuminated Spheros. Once in the HSV color-space a threshold estimating the hue and value of the Sphero's was taken to
produce a binary image. A Gaussian blur was then applied to increase the accuracy of the detector by smoothing the image and
eliminating noise.

After the aforementioned pre-processing the frames were then passed into a Maximally Stable Extremal Regions (MSER) blob
detector. Non-maximal suppression was then computed over the resulting feature keypoints, eliminating any keypoints whose
centers lied within the range of a larger neighboring keypoint. All of the remaining keypoints were considered possible detections
by our algorithm. Though as can be seen in Figure TODO, not all of these detections corresponded to Spheros; reflections on
high gloss floors were particularly challenging for the chosen detector. To limit the effect of such spurious results, we computed a
confidence value for each of our detections. The level of confidence was determined using the range of gray levels that
elucidated an MSER response along with the frequency of localized responses. The confidence in each detection was not used
by the detection algorithm, but was later used as a prior probability of the tracking algorithm's Markov Chain.

The application of another detection algorithm, the Hough Circle transform, was also tested. However, we found that the MSER
blob detector provided a few important advantages. Foremost, the estimate of the Sphero size was found to be closer when
using the MSER blob detector. Extra importance was placed on this factor as the significance of size accuracy was amplified by
the applied non-maximal suppression. Detections of the correct size when using the Hough Circle transform were often
erroneously eliminated by a larger overlapping detection. This issue arises primarily from the Hough Circle transform's reliance
upon the strength of the Canny edge response, which was found to be inconsistent for the illuminated Spheros. The MSER blob
detector, on the other hand, is localized to the center of the response, and offered greater consistency between frames. The
MSER blob detector provided the additional advantage of an easy to implement measure of our confidence in the results of the
detection.

\subsection{Blob Detection Node}
To use the blob detector in the framework, it had to conform to the \emph{observation node} API the framework defines. This API
consists of the functions \emph{preprocess}, \emph{getDetections}, and \emph{getConfidence}. The original node would read
results from a file created by the MATLAB preprocessor, and saves the detections and the confidence of each one.  These results
are then simply returned when either \emph{getDetections} or \emph{getConfidence} are called. To try and mirror this behavior
our blob detection node finds detections and saves its confidence in each one when \emph{preprocess} is called.

\subsection{Parameter Tuning}
The final aspect of our implementation, we found, was a significant amount of tuning. We had previously used the original
framework with a deformable parts model car tracker that used the same observation node as the pedestrian detector. From this
experience, we anticipated the significant amount of tuning of the runtime parameters required to adapt the system to, not only
tracking Spheros instead of pedestrians, but also to the new scene.

First and foremost, the system rejects any detections that would project to an object outside of a certain size range. Our first
calibration step was then to adjust this range from that of an upright adult human to that of a robot the size and shape of a
baseball. Another significant tuning issue was the probability with which a new target exits or enters the scene. The RJ-MCMC
model uses these two parameters, which must be set by hand, when considering whether to initialize a new target to be tracked
and when deciding if a target is no longer in frame. While a significant