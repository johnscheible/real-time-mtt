\section{Discussion}
\label{sec:disc}

\subsection{Further Speed Improvements}
	While our new modified system does not run in real-time, we've shown that its possible for a simple detector to be
	integrated with the system at minimal impact to run time, with the sampling step still the largest bottleneck. Running
	in around 15 minutes, a speedup of 10x is required. Aside from general optimizations of the code, parallelization stands
	out as promising area to begin exploring. First of all, OpenCV has a GPU-enabled, optimized version which could be used,
	presumably offering some not insignificant speedup considering the tracker's reliance on OpenCV code. Additionally, and
	we believe more interesting, is the prospect of parallelizing the MCMC sampling. Unlike detecting potential targets in
	successive frames, MCMC sampling is not inherently required to be done in serial. Given that MCMC sampling is the
	current bottleneck, this could prove to be a massive win for performance. 

\subsection{Parameter Tuning}
	Our hand-on experience with the system has made another issue beyond performance painfully clear. The system can be 
	extremely susceptible to incorrect initialization and requires careful tuning to get good tracking results. This limits the
	portability of the system as someone with rather intimate knowledge needs to tune the parameters for every change from
	a new target type to something as simple as a slightly more crowded dataset from the same scene. While better
	documentation and possibly a better front end could help alleviate this problem, it does not change the fact that for more
	widespread use there are still too many variables that require hand-tuning. 

\subsection{RJ-MCMC}
	After breaking down the various components of the modified tracking system, the RJ-MCMC algorithm was found to
	be the primary cause of the slowdown. Particularly, the number of iterations of random sampling that were required for
	the Markov Chain to produce results dominated by a reasonable approximation of the posterior probability density
	function, the mixing time, was found to be quite large. This was expected given the dimensionality of our observable
	parameters, and the unbounded number of targets to track. However, these factors resulted in a need for more
	compute power to achieve real-time operation using the RJ-MCMC. While the speed-ups mentioned above are
	capable of providing some improvement, in order to achieve a performance gain orders of magnitude better the
	efficiency of the algorithm at the core of this model would need to be improved.