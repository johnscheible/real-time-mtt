\section{Evaluation}
\label{sec:eval}

This section contains a comparative analysis of the original tracking framework and our proposed tracking framework with
respect to two primary categories: time-performance and tracking accuracy.

Time performance was particularly important to us as we were concerned about the effects increasingly spaced frames, due to a
not quite real time tracker, might have on the tracking results.  Furthermore, we were interested in the extensibility of the original
tracking framework to future advances in computer vision, such as improved algorithms with faster runtime.  In order to compare
time performance, we profiled both the original system and our proposed system using Zoom Profiler \cite{zoom2013} on an Intel
3.40 GHzCore i7-2600 running Ubuntu 12.04 with 16 GB of RAM and non-GPU enabled version of OpenCV

\subsection{Original System}
	Our initial assumption was that the major bottleneck of the tracking algorithm was the Deformable Parts Model pedestrian
	detector implemented in MATLAB.  As  Figure~\ref{fig:originalruntime} shows, profiling proved this assumption correct; the
	pre-processing step achieved an average frame-rate of only .44 frames per second.  However, we also found that the
	current tracking framework ran slower than we had initially anticipated; achieving only 1.34 frames per second.  To better
	understand why the tracking algorithm was running so slowly we investigated which functions were consuming the majority
	of the time.  We found that our MCMC sampling function was consuming 54.2\% of the runtime, and so further broke this
	function down as shown in Figure~\ref{fig:runmcmcloopup}.  This revealed that a great deal of time was being spent
	computing the posterior probability distribution using the Markov Chain.  While we did not make any modifications to MCMC
	during our testing, this is one area of further research.
	
	\begin{figure}[!ht]
		\centering
		\begin{tabularx}{0.45\textwidth}{ | X | c | c |}
			\hline
			\textbf{Stage}					&	\textbf{Time}	& 	\textbf{FPS} \\ \hline
			Pre-Processing (MATLAB)		&	45 min		&	0.44 \\ \hline
			Tracking Algorithm (Native)		& 	15 min		&	1.34 \\ \hline
			Total							&	60 min		& 	0.34 \\ \hline
		\end{tabularx}
		\caption{Breakdown of original system runtime}
		\label{fig:originalruntime}
	\end{figure}
	
	\begin{figure}[!htm]
		\centering
		\begin{subfigure}[b]{0.4\textwidth}
			\includegraphics[width=\textwidth]{img/runmcmcbreakdown}
		\end{subfigure}

		\begin{tabularx}{0.45\textwidth}{ | c | X |}
			\hline
			1	&	\emph{\tiny PriorDistCameraEstimate::computeNewSampleMotionPrior()} \\ \hline
			2	&	\emph{\tiny ObservationWrapper::computeNewSampleObservationLkhood()} \\ \hline
			3	&	\emph{\tiny MCMCSample::setNewSample()} \\ \hline
			4	&	\emph{\tiny PriorDistCameraEstimate::ComputeLogMotionPrior()} \\ \hline
			5	&	\emph{\tiny RJMCMCProposal::drawNewSample()} \\ \hline
			6	&	\emph{\tiny ObservationWrapper::computeLogOvservationLkhood()} \\ \hline
			7	&	\emph{\tiny PoseteriorDist::getMeanCamera()} \\ \hline
			8	&	\emph{\tiny MeanShiftWrapper::computeNewSamplesMSLkhood()} \\ \hline
			9	&	\emph{\tiny RJMCMCProposal::dbgCheckCache()} \\ \hline
		\end{tabularx}
		\caption{Breakdown of \emph{runMCMCSampling} runtime}
		\label{fig:runmcmclookup}
	\end{figure}
	
\subsection{``Real-Time" System}
	By eliminating the pre-processing, we made a significant speed-up in overall time. Total run-time for a set of 1209 frames of
	Sphero data was negligibly higher. Considering that the Sphero dataset was shot at 1080p, much higher than the
	Pedestrian data set, this seems to be acceptable slow-down considering the elimination of all pre-processing. We have
	successfully demonstrated that pre-processing can be eliminated with minimal impact on the actual tracker's run time.