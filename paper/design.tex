\section{Design}
\label{sec:design}

To best evaluate the performance of the system and assess whether a real-time solution was possible, our first step was to have
the entire system run as one process. This meant eliminating the preprocessing step that the original system required. This
preprocessing was a \emph{deformable parts model} pedestrian detector written in MATLAB. This step, as we show in our
evaluation, was the largest bottleneck. The preprocessing detection took approximately three times as long as the tracking, while
processing 4 frames in parallel. Since a real-time tracker must process frames in serial, this was totally unacceptable.

\subsection{MSER Blob Detector}
We chose to simplify the detection problem, searching for small robotic balls, known as Spheros, rather than pedestrians. Taking advantage of the Spheros' glowing lights, we choose to use a MSER blob detector as a replacement for the pedestrian detector.
While not necessarily as interesting a tracking target as humans, focusing on a simpler detector allowed us to evaluate the
performance of the tracker itself rather than being limited by a slow detector. Next we will discuss how the framework allowed 
us to swap out the detectors, and why this helps justify our using a simpler detector for our evaluations.

\subsection{MTT Framework's Observation Nodes}
While not extensively documented, the original system has robust support for using additional \emph{observation nodes}. The
original system only used one, simple node. This node would read in the detections from the results of the MATLAB
preprocessing and return the results for the appropriate frame, as it was being processed. Several other nodes which could
report detections such as faces or skin were also written, but not in use.

Extending the framework, we were able to wrap our blob detection algorithm in a node which could process a frame and provide
detections at the point in the tracking algorithm where the original node would have been parsing the file containing the
preprocessed results. While this operation was blocking, the much faster detector ran with no discernible effect on performance
as we will show later. These reasons collectively were our motivation for choosing a simple detector. The system was built in
such a way that when a real-time pedestrian detector is developed it will be just as simple to wrap in an observation node and
insert into the system.