\section{Introduction}
\label{sec:intro}

The \emph{tracking} problem is the application of context to standard object detection. While \emph{detection}
traditionally looks at whether an object is present in a certain picture, tracking requires making a connection between two
or more sequential images and noting the movement of a detected object, or \emph{target}. This is a trivial task to humans,
as we do it every day for many targets of various shapes and sizes, but computationally this is a very difficult problem.

The ability to track multiple targets in real time has many applications from surveillance to artificial intelligence. Contextually
aware robots would have especial need for robust multi-target tracking to be able to interact with humans who perform the
task so effortlessly. The necessity for tracking to occur in real-time is an especially desirable, yet difficult, goal. While
probabilistic methods have been shown to achieve reasonable tracking results, they take a good deal of pre-processing and as
we have shown still fall short of real-time performance. For many applications, that sort of delay is unacceptable 

We have investigated the feasibility of speeding up one such probabilistic framework as described above. This framework,
developed by Choi et al. \cite{choi2010multiple}, is meant to track multiple pedestrians with a single, ``minimally calibrated"
camera. As mentioned above, this particular framework required an expensive pre-processing the data set which took several
times longer than actually running the tracking. We replaced the pre-processing step, which generated pedestrian detections,
with a simpler blob detector that runs in real-time. By simplifying the largest bottleneck of the system, and integrating our detector
as part of the standalone tracker, we've allowed a more extensive analysis of the actual tracking framework and demonstrated
that as faster detectors are developed, they can be integrated in a rather painless way.

The rest of the paper is structured as follows: Section 2 gives an overview of Choi et al.'s original framework and a discussion of
other tracking solutions. Section 3 details our approach to changing the system, including our new detector and how it was
integrated with the original framework. Section 4 covers the actual implementation of the integrated system and how we changed
the framework to track Sphero robots rather than pedestrians. Section 5 presents our experimental evaluation, both in terms of
run time and accuracy. In section 6 we discuss these results, possible future works, and an evaluation of the viability of
RJ-MCMC based tracking as a whole.